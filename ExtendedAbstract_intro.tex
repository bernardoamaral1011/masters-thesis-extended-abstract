\section{Introduction}
\label{sec:intro}

In the present days, air pollution is a major environmental problem. In 2016, it was responsible for approximately 4.2 million deaths and is currently estimated to be accountable for 25\% of adult deaths from strokes, 43\% from chronic obstructive pulmonary disease and 29\% from lung cancer \cite{WHO2018}. It is also responsible for stunting plant growth, lowering agricultural productivity and for reducing city growth and attractiveness to citizens, slowing their evolution and development \cite{GSMA2018}.

At the time of this work, systems for the measurement of air pollution are placed in urban environments as an investment made in the last two decades addressing the emerging knowledge on air pollution health hazards. These systems are constituted by container-style monitoring stations, which occupy large spaces, have high power consumption and high maintenance and production costs. In the greater Lisbon area, there are only fourteen air pollution monitoring stations. From these, only thirteen measure particulate matter with a diameter smaller than 10 micrometers (PM10) and only four measure particulate matter with a diameter smaller than 2.5 micrometers (PM2.5). The monitoring station’s network completed 72 stations in 2005, in the overall area of Portugal \cite{APA2008}.

The goal of this work is to develop a system to improve the resolution of the visualization of PM10 concentration levels throughout the greater area of Lisbon, with the use of the latest Internet of Things (IoT) technologies available, machine learning and a web application. First, a low-cost portable IoT particulate matter (PM) monitoring system will be assembled. Second, with the help of the Lisbon air quality measures and data from Portuguese Online Database on Air Quality (QualAr) several Spatial Interpolation Models (SIM) will be assessed and compared with Fuzzy Boolean Nets (FBN) in what regards spatial interpolation performance. Finally, a web application that presents live, fine resolution, interpolated data of PM10 concentration in the city of Lisbon will be developed.
