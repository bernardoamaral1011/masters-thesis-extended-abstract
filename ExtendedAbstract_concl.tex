\section{Conclusions}
\label{sec:concl}

The current state of the data available publicly regarding air quality was studied and assessed. Several experiments were made in order to evaluate the application of new technologies which could improve it. The tested PMS5003 low-cost PM sensor did not show promising results for official networks integration, and requires complex calibration in order to be used for interpolation purposes. 

Spatial interpolation algorithms showed a low distance correlation, due to the sparsity of the stations in the city, adding to the importance of the study of lower-cost sensors integration. 

It can be concluded that the developed visualization platform could bring awareness to the citizens in what regards their cities air quality, in high resolution, and constitute a replacement for current government platforms.

Relevant results were obtained regarding the usage and performance of every technology used and applied in this work. Finally, information was gathered for further study of the phenomenon of spatial interpolation of air quality, its measurement and its visualization.

\subsection{Future Work}

In a study using NB-IoT, several tests and an evaluation of the study area coverage should be made previously to its application. Low-cost PM monitoring sensors should be further studied in controlled environments. Ideally, with a  calibration function which could minimize the error to extremely low values, for each specific placement location, this type of sensors could integrate air quality monitoring network for interpolation purposes.

Spatial algorithms tested in this work could be complemented with the addition of other independent variables, besides coordinates, and air dispersion models. FBN implementation could be parallelized, and further studied regarding spatial geographical interpolation with the integration of additional variables to the problem.